
\documentclass[a4paper,titlepage,12pt,oneside]{scrbook}
\usepackage{textcomp,graphicx}
\usepackage[ngerman]{babel}
\usepackage{wrapfig}
\usepackage[hang]{subfigure}
\usepackage[ansinew]{inputenc}
\usepackage{amsmath, amsthm}
\usepackage{amsmath,esint} 
\usepackage{amsfonts}   
\usepackage{amssymb}
\usepackage{hyperref}
\usepackage{dsfont}
\usepackage{slashed}
\usepackage {picins} 
\usepackage{tikz}
\usepackage{tikz-3dplot}
\usepackage{eurosym} 
\usepackage{pgfplots}  
\usepackage{wasysym}
\usepackage[misc]{ifsym} 
\usepackage{fixmath}
\usepackage{romannum}  

\usepackage{array,multirow,graphicx} 

  \usepackage{tfrupee} 
\usepackage{tkz-euclide}
\usetkzobj{all}


\usepackage{epsdice}
 


\usetikzlibrary{3d,calc}
\usetikzlibrary{decorations.pathreplacing,calc} 


\usepackage[toc,page]{appendix}

 


\setlength{\parindent}{0pt}



\setcounter{totalnumber}{8}



\hypersetup{
    colorlinks=false,       % false: boxed links; true: colored links
    linkcolor=black,          % color of internal links
    citecolor=black,        % color of links to bibliography
    filecolor=black,      % color of file links
    urlcolor=green          % color of external links
}

\usepackage{titlesec}
\usetikzlibrary{calc, intersections, snakes}

\titleformat{\chapter}{\bf\LARGE}{\thechapter\quad}{0em}{}
\linespread{1.2}



\usepackage[a4paper,left=2.5cm,right=2.5cm,top=2.5cm,bottom=3cm]{geometry}



\newcommand\kariert[2][0.5cm]{% 
   \begin{tikzpicture}[gray,step=#1]
     \pgfmathtruncatemacro\anzahl{(\linewidth-\pgflinewidth)/#1} % maximale Anzahl K�stchen pro Zeile
     \draw (0,0) rectangle (\anzahl*#1,#2*#1) (0,0) grid (\anzahl*#1,#2*#1);
   \end{tikzpicture} 
}

\newcommand{\liniert}[2][0.5cm]{% 
  \begin{tikzpicture}[gray]
  \path[use as bounding box](0,0)rectangle(\linewidth,-#2*#1-0.5\pgflinewidth); 
   \foreach \n in {1,...,#2}\draw(0 ,-#1*\n )--(\linewidth,-#1*\n ); 
  \end{tikzpicture}}



\newcommand\markangle[6][red]{% [color] {X} {origin} {Y} {mark} {radius}
 % filled circle: red by default
  \begin{scope}
  \path[clip] (#2) -- (#3) -- (#4);
  \fill[color=#1,fill = white,draw=#1,name path=circle] %opacity=0.5
  (#3) circle (#6mm);
   \end{scope}
   % middle calculation
   \path[name path=line one] (#3) -- (#2);
  \path[name path=line two] (#3) -- (#4);
\path[%
name intersections={of=line one and circle, by={inter one}},
 name intersections={of=line two and circle, by={inter two}}
 ] (inter one) -- (inter two) coordinate[pos=.5] (middle);
 % bissectrice definition
 \path[%
name path=bissectrice
] (#3) -- (barycentric cs:#3=-1,middle=1.2);
 % put mark
 \path[
 name intersections={of=bissectrice and circle, by={middleArc}}
 ] (#3) -- (middleArc) node[pos=0.6] {#5}; % node[pos=1.3]
 }
 
 









\newcommand{\tikzmark}[2][-3pt]{\tikz[remember picture, overlay, baseline=-0.5ex]\node[#1](#2){};}

\tikzset{brace/.style={decorate, decoration={brace}},
 brace mirrored/.style={decorate, decoration={brace,mirror}},
}

\newcounter{brace}
\setcounter{brace}{0}
\newcommand{\drawbrace}[3][brace]{%
 \refstepcounter{brace}
 \tikz[remember picture, overlay]\draw[#1] (#2.center)--(#3.center)node[pos=0.5, name=brace-\thebrace]{};
}

\newcounter{arrow}
\setcounter{arrow}{0}
\newcommand{\drawcurvedarrow}[3][]{%
 \refstepcounter{arrow}
 \tikz[remember picture, overlay]\draw (#2.center)edge[#1]node[coordinate,pos=0.5, name=arrow-\thearrow]{}(#3.center);
}

% #1 options, #2 position, #3 text 
\newcommand{\annote}[3][]{%
 \tikz[remember picture, overlay]\node[#1] at (#2) {#3};
}




\usepackage{xparse}
\usepackage{mathtools}

\makeatletter
\newlength\fib@width
\def\fib@widthfactor{1.75}
\newif\iffibhideanswer
\fibhideanswertrue
\tikzset{
   every fill in box/.style={
     inner xsep=0pt,
     minimum height=3ex,
     align=center,
     font={\sffamily\slshape},
   },
   colored box/.style={
      every fill in box,
      fill=blue!35!white,
   },
   framed box/.style={
      every fill in box,
      draw,
   },
   underline style/.style={},
   underlined box/.style={
      every fill in box,
      append after command={%
         \pgfextra{\begin{pgfinterruptpath}
            \draw [underline style] (\tikzlastnode.south west)
                  -- (\tikzlastnode.south east);
         \end{pgfinterruptpath}}
      },
   },
   bracket style/.style={},
   underbracked box/.style={
      every fill in box,
      append after command={%
         \pgfextra{\begin{pgfinterruptpath}
            \draw [bracket style] ($(\tikzlastnode.south west)+(0,2pt)$)
                  |- (\tikzlastnode.south)
                  -| ($(\tikzlastnode.south east)+(0,2pt)$);
         \end{pgfinterruptpath}}
      },
   },
   underoverbracked box/.style={
      every fill in box,
      append after command={%
         \pgfextra{\begin{pgfinterruptpath}
            \draw [bracket style] ($(\tikzlastnode.north west)-(0,2pt)$)
                  |- (\tikzlastnode.north)
                  -| ($(\tikzlastnode.north east)-(0,2pt)$);
            \draw [bracket style] ($(\tikzlastnode.south west)+(0,2pt)$)
                  |- (\tikzlastnode.south)
                  -| ($(\tikzlastnode.south east)+(0,2pt)$);
         \end{pgfinterruptpath}}
      },
   },
   fill in/.style={
      colored box,
   },
}
\NewDocumentCommand { \fib@hide } { m } {%
   \iffibhideanswer
      \phantom{#1}%
   \else
      #1%
   \fi
}
\NewDocumentCommand { \fib@makebox }{ m }{%
   \settowidth{\fib@width}{\tikz\node[fill in]{#1};}%
   \begin{tikzpicture}[baseline=(fill in node.base)]
      \node (fill in node) [text width=\fib@widthfactor*\fib@width,fill in] {%
         \fib@hide{#1}%
      };
   \end{tikzpicture}%
}
\NewDocumentCommand { \fib } { s d{<}{>} o m }{{%
   \IfBooleanT{#1}{\fibhideanswerfalse}%
   \IfValueT{#2}{\only<#2>{\fibhideanswerfalse}}%
   \IfValueT{#3}{\tikzset{fill in/.style={#3}}}%
   \ifmmode
      \mathchoice
         {\fib@makebox{$\displaystyle#4$}}
         {\fib@makebox{$\textstyle#4$}}
         {\fib@makebox{$\scriptstyle#4$}}
         {\fib@makebox{$\scriptscriptstyle#4$}}
   \else
      \fib@makebox{#4}%
   \fi
   \IfValueT{#2}{}%
}}
\makeatother






 \setcounter{tocdepth}{1}
 






% Three counters
\newcounter{xcube}
\newcounter{ycube}
\newcounter{zcube}

% The angles of x,y,z-axes
\newcommand\xaxis{210}
\newcommand\yaxis{-30}
\newcommand\zaxis{90}

% The top side of a cube
\newcommand\topside[3]{
  \fill[fill=green!45, draw=black,shift={(\xaxis:#1)},shift={(\yaxis:#2)},
  shift={(\zaxis:#3)}] (0,0) -- (30:1) -- (0,1) --(150:1)--(0,0);
}

% The left side of a cube
\newcommand\leftside[3]{
  \fill[fill=red!45, draw=black,shift={(\xaxis:#1)},shift={(\yaxis:#2)},
  shift={(\zaxis:#3)}] (0,0) -- (0,-1) -- (210:1) --(150:1)--(0,0);
}

% The right side of a cube
\newcommand\rightside[3]{
  \fill[fill=blue!45, draw=black,shift={(\xaxis:#1)},shift={(\yaxis:#2)},
  shift={(\zaxis:#3)}] (0,0) -- (30:1) -- (-30:1) --(0,-1)--(0,0);
}

% The cube 
\newcommand\cube[3]{
  \topside{#1}{#2}{#3} \leftside{#1}{#2}{#3} \rightside{#1}{#2}{#3}
}

% Definition of \planepartition
% To draw the following plane partition, just write \planepartition{ {a, b, c}, {d,e} }.
%  a b c
%  d e
\newcommand\planepartition[1]{
 \setcounter{xcube}{-1}
  \foreach \a in {#1} {
    \addtocounter{xcube}{1}
    \setcounter{ycube}{-1}
    \foreach \b in \a {
      \addtocounter{ycube}{1}
      \setcounter{zcube}{-1}
      \foreach \c in {1,...,\b} {
        \addtocounter{zcube}{1}
        \cube{\value{xcube}}{\value{ycube}}{\value{zcube}}
      }
    }
  }
}



\usepackage{yfonts} 
 
\usetikzlibrary{calc, intersections}


\usepackage[europeanresistors,cuteinductors,fulldiodes,siunitx]{circuitikz}

  

 
 

\usetikzlibrary{shadows.blur}
\usetikzlibrary{shapes.symbols}

 
 
\usetikzlibrary{arrows.meta}


\begin{document}  


\definecolor{almond}{rgb}{0.94, 0.87, 0.8}
\definecolor{myblue}{HTML}{92dcec} 
\pagenumbering{arabic} 



\def\bitcoinA{%
  \leavevmode
  \vtop{\offinterlineskip %\bfseries
    \setbox0=\hbox{B}%
    \setbox2=\hbox to\wd0{\hfil\hskip-.03em
    \vrule height .3ex width .15ex\hskip .08em
    \vrule height .3ex width .15ex\hfil}
    \vbox{\copy2\box0}\box2}}
        
    \pagestyle{empty}
     
		 
%\begin{minipage}{0.48\textwidth}
%\end{minipage}%
%\hfill 
%\begin{minipage}{0.48\textwidth}
%\end{minipage}%
%\hfill  
		 

%%% , rounded corners=6pt, blur shadow={shadow blur steps=5,shadow blur extra rounding=1.3pt}, text width=3cm

%\begin{tikzpicture}[scale=1,>=latex] 
%\end{tikzpicture} 

%\definecolor{almond}{rgb}{0.94, 0.87, 0.8}

%\scalebox{.5}{}      to[out=290,in=110]     \rule{0pt}{20pt} 

%  \begin{scope}[yshift=-0.25cm,xshift=-0.25cm] \shade[ball color=red!100!white, opacity=0.99] (0,0) circle (0.2cm) node {$+$};  \end{scope}    ,shorten >=\z

%   \begin{align*}     \end{align*} 


 
\begin{table*}[htbp]
\centering
    \begin{tabular}{ c  c  c  }
     IQSH-Handreichung & \qquad\qquad\qquad\qquad\qquad\qquad\qquad  \qquad\qquad\qquad\quad  & Prozentrechnung 		 \\ \hline\hline
    \end{tabular}
\end{table*} \vspace{-1.25em}

\begin{center}
\Large \textbf{- Erarbeitung von prozentualen Anteilen - L�sungen -} \\
\end{center} 

\vspace{0.75em}


\begin{minipage}{0.04\textwidth}
\includegraphics[width=0.99 \textwidth]{Work.png}   \vspace{-0.5em}
\end{minipage}%
\hfill 
\begin{minipage}{0.95\textwidth}
\textbf{Aufgabe 1:} Gib die Zahlen in der Prozentdarstellung an.
\end{minipage}%   


 \begin{align*}
& a) \;\; 0,06= 6\%  	\qquad\qquad\qquad\,	&&b) \;\; 0,96= 96\%   \qquad\qquad\qquad\;\;	&&c) \;\; 0,7  =  70\%    \qquad\qquad\qquad \\
& d) \;\; \frac{1}{4}= 25\%    	\qquad\qquad\;\;\;\;\;\,	&&e) \;\; \frac{17}{20} = 85\%   \qquad\qquad\quad\,	&&f) \;\; \frac{3}{5}= 60\%   \qquad\qquad\qquad \\
& g) \;\; 0,6541 = 65,41\%   	\qquad\qquad\quad  	&&h) \;\; 7  = 700\%  \qquad\qquad\quad\;	&&i) \;\; \frac{11}{9}  = 1,\bar{2}\% \qquad\qquad\qquad  
\end{align*} 
 
\vspace{2em}


\begin{minipage}{0.04\textwidth}
\includegraphics[width=0.99 \textwidth]{Work.png}   \vspace{-0.5em}
\end{minipage}%
\hfill 
\begin{minipage}{0.95\textwidth}
\textbf{Aufgabe 2:} Gib die Zahlen in der gek�rzten Bruch- und Dezimalzahldarstellung an.
\end{minipage}%   


 \begin{align*}
& a) \;\; 30\%=0,3=\frac{3}{10}   	\qquad\qquad\;\;	&&b) \;\; 6,25\%=0,0625=\frac{1}{16}    \qquad\;\;\\
&c) \;\; 44,\bar{4}\%  = 0,\bar{4} = \frac{4}{9}  \qquad\qquad\;\;	&&d) \;\; 125\%  = 1,25 = \frac{5}{4}      \qquad\qquad\qquad\qquad  
\end{align*} 


\vspace{2em}




\begin{minipage}{0.1\textwidth}
\includegraphics[width=0.4 \textwidth]{Kompass.png} 
\includegraphics[width=0.4 \textwidth]{Work.png}   \vspace{1em}
\end{minipage}%
\hfill 
\begin{minipage}{0.89\textwidth}
\textbf{Aufgabe 3:} Gib den dargestellten roten Anteil vom Ganzen in der Prozentdarstellung an.
\end{minipage}%   

 

\begin{center}
\begin{tikzpicture}[scale=1,>=latex]

\begin{scope}[xshift=0cm]
\draw[thick,fill=red,opacity=0.5] (0,0) -- (0:1)  arc (0:90:1) -- (0,0); 
\draw[thick,fill=red,opacity=0.5] (0,0) -- (180:1)  arc (180:225:1) -- (0,0); 
\foreach \n in {45,90,...,360} { \draw[opacity=0.5,rotate=\n] (0,0) -- (0:1)  arc (0:45:1) -- (0,0); } 
\draw[thick] (0,0) circle (1cm); 
\node at (-1.25,1) {$a)$};	
\node at (0,-1.35) {$37,5\%$};	
\end{scope}

\begin{scope}[xshift=3cm]
\draw[thick,fill=red,opacity=0.5] (0,0) -- (270:1)  arc (270:330:1) -- (0,0); 
\draw[thick,fill=red,opacity=0.5] (0,0) -- (180:1)  arc (180:210:1) -- (0,0); 
\draw[thick,fill=red,opacity=0.5] (0,0) -- (90:1)  arc (90:150:1) -- (0,0); 
\foreach \n in {30,60,...,360} { \draw[opacity=0.5,rotate=\n] (0,0) -- (0:1)  arc (0:30:1) -- (0,0); } 
\draw[thick] (0,0) circle (1cm); 
\node at (-1.25,1) {$b)$};	
\node at (0,-1.35) {$41,\bar{6}\%$};	
\end{scope}

\begin{scope}[xshift=6cm]
\draw[thick,fill=white,opacity=0.5] (0,0) rectangle (0.5,0.5);  
\draw[thick,fill=red,opacity=0.5] (0,0.5) rectangle (.5,1);  
\draw[thick,fill=white,opacity=0.5] (0.5,0) rectangle (1,.5);  
\draw[thick,fill=red,opacity=0.5] (0.5,0.5) rectangle (1,1);   
\begin{scope}[xshift=-1cm]
\draw[thick,fill=white,opacity=0.5] (0,0) rectangle (0.5,0.5);  
\draw[thick,fill=white,opacity=0.5] (0,0.5) rectangle (.5,1);  
\draw[thick,fill=white,opacity=0.5] (0.5,0) rectangle (1,.5);  
\draw[thick,fill=white,opacity=0.5] (0.5,0.5) rectangle (1,1);
\end{scope}   
\begin{scope}[yshift=-1cm]
\draw[thick,fill=red,opacity=0.5] (0,0) rectangle (0.5,0.5);  
\draw[thick,fill=white,opacity=0.5] (0,0.5) rectangle (.5,1);  
\draw[thick,fill=white,opacity=0.5] (0.5,0) rectangle (1,.5);  
\draw[thick,fill=red,opacity=0.5] (0.5,0.5) rectangle (1,1);   
\begin{scope}[xshift=-1cm]
\draw[thick,fill=red,opacity=0.5] (0,0) rectangle (0.5,0.5);  
\draw[thick,fill=white,opacity=0.5] (0,0.5) rectangle (.5,1);  
\draw[thick,fill=white,opacity=0.5] (0.5,0) rectangle (1,.5);  
\draw[thick,fill=white,opacity=0.5] (0.5,0.5) rectangle (1,1);
\end{scope} \end{scope}   
\node at (-1.25,1) {$c)$};	
\node at (0,-1.35) {$31,25\%$};	
\end{scope}

\begin{scope}[xshift=9cm]
\draw[thick,fill=white,opacity=0.5] (0,0) rectangle (0.5,0.5);  
\draw[thick,fill=red,opacity=0.5] (0,0.5) rectangle (.5,1);  
\draw[thick,fill=white,opacity=0.5] (0.5,0) rectangle (1,.5);  
\draw[thick,fill=red,opacity=0.5] (0.5,0.5) rectangle (1,1);   
\begin{scope}[xshift=-1cm]
\draw[thick,fill=white,opacity=0.5] (0,0) rectangle (0.5,0.5);  
\draw[thick,fill=red,opacity=0.5] (0,0.5) rectangle (.5,1);  
\draw[thick,fill=white,opacity=0.5] (0.5,0) rectangle (1,.5);  
\draw[thick,fill=white,opacity=0.5] (0.5,0.5) rectangle (1,1);
\end{scope}     
\begin{scope}[xshift=1cm]
\draw[thick,fill=white,opacity=0.5] (0,0) rectangle (0.5,0.5);  
\draw[thick,fill=white,opacity=0.5] (0,0.5) rectangle (.5,1);   
\end{scope}   
\begin{scope}[yshift=-1cm]
\draw[thick,fill=red,opacity=0.5] (0,0) rectangle (0.5,0.5);  
\draw[thick,fill=white,opacity=0.5] (0,0.5) rectangle (.5,1);  
\draw[thick,fill=white,opacity=0.5] (0.5,0) rectangle (1,.5);  
\draw[thick,fill=red,opacity=0.5] (0.5,0.5) rectangle (1,1);   
\begin{scope}[xshift=-1cm]
\draw[thick,fill=red,opacity=0.5] (0,0) rectangle (0.5,0.5);  
\draw[thick,fill=white,opacity=0.5] (0,0.5) rectangle (.5,1);  
\draw[thick,fill=red,opacity=0.5] (0.5,0) rectangle (1,.5);  
\draw[thick,fill=white,opacity=0.5] (0.5,0.5) rectangle (1,1);
\end{scope}   
\begin{scope}[xshift=1cm]
\draw[thick,fill=red,opacity=0.5] (0,0) rectangle (0.5,0.5);  
\draw[thick,fill=red,opacity=0.5] (0,0.5) rectangle (.5,1);   
\end{scope} \end{scope}   
\node at (-1.25,1) {$d)$};	
\node at (0,-1.35) {$45\%$};	
\end{scope}



\begin{scope}[xshift=11.5cm]\begin{scope}[yshift=-1cm]
\draw[thick,fill=white,opacity=0.5] (0,0) rectangle (0.5,2); 
\draw[thick,fill=red,opacity=0.5] (0.5,0) rectangle (1,2); 
\draw[thick,fill=white,opacity=0.5] (1,0) rectangle (1.5,2); 
\draw[thick,fill=white,opacity=0.5] (1.5,0) rectangle (2,2); 
\draw[thick,fill=red,opacity=0.5] (2,0) rectangle (2.5,2); 
\draw[thick,fill=white,opacity=0.5] (2.5,0) rectangle (3,2); 
\draw[thick,fill=white,opacity=0.5] (3,0) rectangle (3.5,2); 
\end{scope}      
\node at (-0.25,1) {$e)$};	
\node at (0,-1.35) {$28,\overline{571428}\%$};	\end{scope}

\end{tikzpicture}
\end{center}



\vspace{2em}


\begin{minipage}{0.1\textwidth}
\includegraphics[width=0.4 \textwidth]{Kompass.png} 
\includegraphics[width=0.4 \textwidth]{Work.png}   \vspace{1em}
\end{minipage}%
\hfill 
\begin{minipage}{0.89\textwidth}
\textbf{Aufgabe 4:} Beantworte die Fragen und beschreibe die Auff�lligkeit.
\end{minipage}%   



 \begin{align*}
 &a) \;\;  \begin{array}{l} \text{Wie viel sind $10\%$ von $50$? $\;\Rightarrow\; 5$}  \\ \text{Wie viel sind $50\%$ von $10$? $\;\Rightarrow\; 5$}   \end{array} \qquad 
&&b) \;\;  \begin{array}{l} \text{Wie viel sind $17\%$ von $100$? $\;\Rightarrow\; 17$}  \\ \text{Wie viel sind $100\%$ von $17$? $\;\Rightarrow\; 17$}   \end{array}  \\  
 &c) \;\;  \begin{array}{l} \text{Wie viel sind $8\%$ von $25$? $\;\Rightarrow\; 2$}  \\ \text{Wie viel sind $25\%$ von $8$? $\;\Rightarrow\; 2$}   \end{array}  \qquad 
&&d) \;\;  \begin{array}{l} \text{Wie viel sind $7\%$ von $800$? $\;\Rightarrow\; 42$}  \\ \text{Wie viel sind $800\%$ von $7$? $\;\Rightarrow\; 42$}   \end{array}    \qquad \qquad \qquad \qquad \qquad \qquad \qquad \qquad \qquad \qquad \qquad \qquad 
\end{align*} 
\vspace{1em}

Da es sich bei dem "`von"'-Prinzip um eine Multiplikation mit Br�chen handelt, ist es gleich an welcher Zahl das Prozentzeichen angeheftet wird.

 

\newpage

 
\begin{table*}[htbp]
\centering
    \begin{tabular}{ c  c  c  }
     IQSH-Handreichung & \qquad\qquad\qquad\qquad\qquad\qquad\qquad  \qquad\qquad\qquad\quad  & Prozentrechnung 		 \\ \hline\hline
    \end{tabular}
\end{table*} \vspace{-1.25em}

\begin{center}
\Large \textbf{- Erarbeitung von prozentualen Anteilen - L�sungen -} \\
\end{center} 

\vspace{0.75em}



\begin{minipage}{0.1\textwidth}
\includegraphics[width=0.4 \textwidth]{Kompass.png} 
\includegraphics[width=0.4 \textwidth]{Work.png}   \vspace{1em}
\end{minipage}%
\hfill 
\begin{minipage}{0.89\textwidth}
\textbf{Aufgabe 5:} Beantworte die Fragen und beschreibe die Auff�lligkeit.
\end{minipage}%   



$a)\; \begin{array}{l} \text{Wie viel sind $\frac{1}{2}$ von $100$?} \;\Rightarrow\; 50  \\ \text{Wie viel sind $\frac{1}{2}$ von $50$? } \;\Rightarrow\; 25 \\  \text{Wie viel sind $\frac{1}{2}$ von $\frac{1}{2}$ von $100$?}  \;\Rightarrow\; 25 \end{array}$ \quad   $b)\; \begin{array}{l} \text{Wie viel sind $30\%$ von $100$?}  \;\Rightarrow\; 30\\ \text{Wie viel sind $50\%$ von $30$?}  \;\Rightarrow\; 15 \\  \text{Wie viel sind $30\%$ von $50\%$ von $100$? } \;\Rightarrow\; 15 \end{array}$   \\

\vspace{0.8em}

$c)\; \begin{array}{l} \text{Wie viel sind $75\%$ von $100\%$?} \;\Rightarrow\; 75 \\ \text{Wie viel sind $66,\bar{6}\%$ von $75\%$?} \;\Rightarrow\; 50  \\  \text{Wie viel sind $20\%$ von $50\%$?} \;\Rightarrow\; 10 \\  \text{Wie viel sind $75\%$ von $66,\bar{6}\%$ von $20\%$ von $100\%$?} \;\Rightarrow\; 10  \end{array}$  \\

\vspace{1em}

Ob die Anteile nach und nach ausgerechnet werden oder alle auf einmal verrechnet werden, macht keinen Unterschied, da die Multiplikation kommutativ ist. Bei Anteilen von Anteilen handelt es sich somit lediglich um eine Multiplikation von Br�chen.

 


% \begin{align*}   \end{align*} 


\vspace{2em}




\begin{minipage}{0.1\textwidth}
\includegraphics[width=0.4 \textwidth]{Kompass.png} 
\includegraphics[width=0.4 \textwidth]{Work.png}   \vspace{1em}
\end{minipage}%
\hfill 
\begin{minipage}{0.89\textwidth}
\textbf{Aufgabe 6:} F�lle die L�cken im L�ckentext so aus, dass die beiden Aussagen zusammen stimmen.
\end{minipage}%   

 
\vspace{1em}
		
$a)\;\;$ Ein ganzer Kreis hat $360^\circ$, somit hat ein Sechstel Kreis $\fib*{60^\circ}$.  \\
$b)\;\;$ $180^\circ$ bei einem Kreis entsprechen $50\%$ vom Kreis. $\fib*{3,6^\circ}$ entsprechen $1\%$ von einem Kreis.   \\
$c)\;\;$ $2000$\EUR{} entsprechen $100\%$, dann entsprechen $\fib*{20\text{\EUR{}}}$ $1\%$.   \\
$d)\;\;$ Wenn $400\,$m als $20\%$ verstanden werden, dann sind $\fib*{2000\,\text{m}}$ als $100\%$ zu verstehen.  \\		
		
		
 


		 
\vspace{3em}
 


\begin{minipage}{0.1\textwidth}
\includegraphics[width=0.4 \textwidth]{Work.png} 
\includegraphics[width=0.4 \textwidth]{Puzzle.png}   \vspace{1em}
\end{minipage}%
\hfill 
\begin{minipage}{0.89\textwidth}
\textbf{Aufgabe 7:} F�lle die L�cken im L�ckentext so aus, dass die Aussage auf die Skizze zu trifft.
\end{minipage}%   


\vspace{1em}



\begin{minipage}{0.25\textwidth}
\begin{center}
\begin{tikzpicture}[scale=1,>=latex]
 
\begin{scope}[xshift=0cm]
\draw[thick,fill=white,opacity=0.5] (0,0) rectangle (0.5,0.5);  
\draw[thick,fill=white,opacity=0.5] (0,0.5) rectangle (.5,1);  
\draw[thick,fill=white,opacity=0.5] (0.5,0) rectangle (1,.5);  
\draw[thick,fill=white,opacity=0.5] (0.5,0.5) rectangle (1,1);   
\begin{scope}[xshift=-1cm]
\draw[thick,fill=white,opacity=0.5] (0,0) rectangle (0.5,0.5);  
\draw[thick,fill=white,opacity=0.5] (0,0.5) rectangle (.5,1);  
\draw[thick,fill=white,opacity=0.5] (0.5,0) rectangle (1,.5);  
\draw[thick,fill=white,opacity=0.5] (0.5,0.5) rectangle (1,1);
\end{scope}     
\begin{scope}[xshift=1cm]
\draw[thick,fill=white,opacity=0.5] (0,0) rectangle (0.5,0.5);  
\draw[thick,fill=white,opacity=0.5] (0,0.5) rectangle (.5,1);   
\end{scope}   
\begin{scope}[yshift=-1cm]
\draw[thick,fill=white,opacity=0.5] (0,0) rectangle (0.5,0.5);  
\draw[thick,fill=white,opacity=0.5] (0,0.5) rectangle (.5,1);  
\draw[thick,fill=white,opacity=0.5] (0.5,0) rectangle (1,.5);  
\draw[thick,fill=white,opacity=0.5] (0.5,0.5) rectangle (1,1);   
\begin{scope}[xshift=-1cm]
\draw[thick,fill=white,opacity=0.5] (0,0) rectangle (0.5,0.5);  
\draw[thick,fill=white,opacity=0.5] (0,0.5) rectangle (.5,1);  
\draw[thick,fill=white,opacity=0.5] (0.5,0) rectangle (1,.5);  
\draw[thick,fill=white,opacity=0.5] (0.5,0.5) rectangle (1,1);
\end{scope}   
\begin{scope}[xshift=1cm]
\draw[thick,fill=white,opacity=0.5] (0,0) rectangle (0.5,0.5);  
\draw[thick,fill=white,opacity=0.5] (0,0.5) rectangle (.5,1);   
\end{scope} \end{scope}   
\node at (-1.25,1) {$a)$};	
\end{scope}
\draw[thick,fill=red,opacity=0.5] (0.5,-1) rectangle (1.5,1); 
\draw[thick,fill=blue,opacity=0.5] (-1,-1) rectangle (1.5,-0.5); 

\end{tikzpicture}
\end{center}
\end{minipage}%  
\begin{minipage}{0.74\textwidth} 

$40\%$ vom Rechteck sind $\fib*{\text{rot}}$ markiert. $25\%$ vom Rechteck sind $\fib*{\text{blau}}$ markiert. $\fib*{10\%}$ vom Rechteck sind rot und blau markiert.

\end{minipage}%    

\vspace{1.5em}

\begin{minipage}{0.25\textwidth} 

\begin{center}
\begin{tikzpicture}[scale=1,>=latex]


\begin{scope}[xshift=0cm]
\draw[thick,fill=white,opacity=0.5] (0,0) rectangle (0.5,0.5);  
\draw[thick,fill=white,opacity=0.5] (0,0.5) rectangle (.5,1);     
\begin{scope}[xshift=-1cm]
\draw[thick,fill=white,opacity=0.5] (0,0) rectangle (0.5,0.5);  
\draw[thick,fill=white,opacity=0.5] (0,0.5) rectangle (.5,1);  
\draw[thick,fill=white,opacity=0.5] (0.5,0) rectangle (1,.5);  
\draw[thick,fill=white,opacity=0.5] (0.5,0.5) rectangle (1,1);
\end{scope}    
\begin{scope}[yshift=-1cm]
\draw[thick,fill=white,opacity=0.5] (0,0) rectangle (0.5,0.5);  
\draw[thick,fill=white,opacity=0.5] (0,0.5) rectangle (.5,1);    
\begin{scope}[xshift=-1cm]
\draw[thick,fill=white,opacity=0.5] (0,0) rectangle (0.5,0.5);  
\draw[thick,fill=white,opacity=0.5] (0,0.5) rectangle (.5,1);  
\draw[thick,fill=white,opacity=0.5] (0.5,0) rectangle (1,.5);  
\draw[thick,fill=white,opacity=0.5] (0.5,0.5) rectangle (1,1);
\end{scope}      
\end{scope}  
\node at (-1.25,1) {$b)$};	
\end{scope}
\draw[thick,fill=red,opacity=0.5] (-1,-1) rectangle (0,1); 
\draw[thick,fill=blue,opacity=0.5] (-1,-1) rectangle (0.5,-0.5); 

\end{tikzpicture}
\end{center}

\end{minipage}%    
\begin{minipage}{0.74\textwidth} 


$25\%$ vom Rechteck sind $\fib*{\text{blau}}$ markiert. $66,\bar{6}\%$ vom Rechteck sind $\fib*{\text{rot}}$ markiert. $\fib*{66,\bar{6}\%}$ von den blau markierten sind auch rot markiert, also sind $\fib*{16,\bar{6}\%}$ vom Rechteck rot und blau markiert.

\end{minipage}%    
   
 
\vspace{3em}




\newpage

 
\begin{table*}[htbp]
\centering
    \begin{tabular}{ c  c  c  }
     IQSH-Handreichung & \qquad\qquad\qquad\qquad\qquad\qquad\qquad  \qquad\qquad\qquad\quad  & Prozentrechnung 		 \\ \hline\hline
    \end{tabular}
\end{table*} \vspace{-1.25em}

\begin{center}
\Large \textbf{- Erarbeitung von prozentualen Anteilen - L�sungen -} \\
\end{center} 

\vspace{0.75em}



\begin{minipage}{0.1\textwidth}
\includegraphics[width=0.4 \textwidth]{Kompass.png} 
\includegraphics[width=0.4 \textwidth]{Work.png}   \vspace{5em}
\end{minipage}%
\hfill 
\begin{minipage}{0.89\textwidth}
\textbf{Aufgabe 8:} Bei einer Umfrage gaben $16,7\%$ der Personen die Antwort $G$, $8,3\%$ die Antwort $K$,  $37,5\%$ die Antwort $F$,  $25\%$ die Antwort $B$ und  $9,7\%$ die Antwort $N$ an. Gib an, welches Kreisdiagramm den Sachzusammenhang richtig darstellt und begr�nde, warum die anderen beiden Optionen nicht richtig sein k�nnen.
\end{minipage}%   


\begin{minipage}{0.33\textwidth}

\begin{center}
\begin{tikzpicture}[scale=0.9]
    \foreach \start/\end/\middle/\percent/\anchor		/\color		/\name			/\smeter in 	{
							0			/36		/30			/25{,}0		 /above			/blue!		/a 						/1 				,
							36		/130	/105		/32{,}0		/above			/red!			/b						/1 						,
							130		/230	/165		/15{,}5		/below			/green!		/c						/1 						,
							230		/280	/255		/10{,}5		/right			/orange!	/d						/1 						,
							280		/360	/340  	/17{,}0		/right			/pink!		/e						/1 						}
  {
    \draw[fill=\color , opacity=0.25 , thick] (0,0) -- (\end:2.4cm) arc (\end:\start:2.4cm)
      node[opacity=0.99] at (\middle:1.85cm) { }; 
		\node (A) at (\middle:2.99cm)   {$$};
  };
	\end{tikzpicture}
\end{center}


\end{minipage}%
\begin{minipage}{0.33\textwidth}

\begin{center}
\begin{tikzpicture}[scale=0.9]
    \foreach \start/\end/\middle/\percent/\anchor		/\color		/\name			/\smeter in 	{
							0			/60		/30			/25{,}0		 /above			/blue!		/a 						/1 				,
							60		/150	/105		/32{,}0		/above			/red!			/b						/1 						,
							150		/180	/165		/15{,}5		/below			/green!		/c						/1 						,
							180		/315	/255		/10{,}5		/right			/orange!	/d						/1 						,
							315		/360	/340  	/17{,}0		/right			/pink!		/e						/1 						}
  {
    \draw[fill=\color , opacity=0.25 , thick] (0,0) -- (\end:2.4cm) arc (\end:\start:2.4cm)
      node[opacity=0.99] at (\middle:1.85cm) { }; 
		\node (A) at (\middle:2.99cm)   {$$};
  };
	\end{tikzpicture}
\end{center}


\end{minipage}%
\begin{minipage}{0.33\textwidth}

\begin{center}
\begin{tikzpicture}[scale=0.9]
    \foreach \start/\end/\middle/\percent/\anchor		/\color		/\name			/\smeter in 	{
							0			/160		/30			/25{,}0		 /above			/blue!		/a 						/1 				,
							160		/200	/105		/32{,}0		/above			/red!			/b						/1 						,
							200		/270	/165		/15{,}5		/below			/green!		/c						/1 						,
							270		/300	/255		/10{,}5		/right			/orange!	/d						/1 						,
							300		/360	/340  	/17{,}0		/right			/pink!		/e						/1 						}
  {
    \draw[fill=\color , opacity=0.25 , thick] (0,0) -- (\end:2.4cm) arc (\end:\start:2.4cm)
      node[opacity=0.99] at (\middle:1.85cm) { }; 
		\node (A) at (\middle:2.99cm)   {$$};
  };
	\end{tikzpicture}
\end{center}


\end{minipage}%


\vspace{2em}

Es muss das Diagramm in der Mitte sein, da das linke Diagramm keinen Anteil deutlich �ber $90^\circ$ besitzt, was den $37,5\%$ entsprechen k�nnte. Beim rechten Diagramm fehlt ein Winkel von $90^\circ$, der der $25\%$ verk�rpert.


\newpage

 
\begin{table*}[htbp]
\centering
    \begin{tabular}{ c  c  c  }
     IQSH-Handreichung & \qquad\qquad\qquad\qquad\qquad\qquad\qquad  \qquad\qquad\qquad\quad  & Prozentrechnung 		 \\ \hline\hline
    \end{tabular}
\end{table*} \vspace{-1.25em}

\begin{center}
\Large \textbf{- Erarbeitung von prozentualen Anteilen - L�sungen -} \\
\end{center} 

\vspace{0.75em}



\begin{minipage}{0.04\textwidth}
\includegraphics[width=0.99 \textwidth]{Puzzle.png}    \vspace{2em}
\end{minipage}%
\hfill 
\begin{minipage}{0.95\textwidth}
\textbf{Aufgabe 9:} Lies die Behauptungen zum Diagramm zur Darstellung einer Wahl im Vergleich zur letzten Wahl und kreuze wahr, falsch oder nicht bestimmbar an. Begr�nde deine Wahl. 
\end{minipage}%   


\begin{center}
\begin{tikzpicture}[scale=1.0,>=latex]
 
  \coordinate (o) at (0,0);
  \coordinate (y) at (0,5.25);
    \coordinate (x) at (10.25,0);
    \draw[<->, black!100, thick] (y) node[above] {Stimmenanteil in $\%$} -- (0,0) --  (x) node[right]
    {Partei};

\draw[-, black!100, thin]  (0,0.1) -- (0,-0.1) node[below=0.25cm,left] {0};
\draw[-, black!100, thin]  (1.9,0.1) -- (1.9,-0.1) node[below] {$A$};
\draw[-, black!100, thin]  (4.9,0.1) -- (4.9,-0.1) node[below] {$B$};
\draw[-, black!100, thin]  (7.9,0.1) -- (7.9,-0.1) node[below] {$C$}; 
\draw[-, black!100, thin]  (10,1) -- (-0.1,1) node[left] {$$};
\draw[-, black!100, thin]  (10,3) -- (-0.1,3) node[left] {$$};
\draw[-, black!100, thin]  (10,5) -- (-0.1,5) node[left] {$$};
\draw[-, black!100, thin,densely dotted]  (10,1.5) -- (-0.1,1.5) node[left] {$$};
\draw[-, black!100, thin,densely dotted]  (10,3.5) -- (-0.1,3.5) node[left] {$$}; 
\draw[-, black!100, thin,densely dotted]  (10,2.5) -- (-0.1,2.5) node[left] {$$};
\draw[-, black!100, thin,densely dotted]  (10,4.5) -- (-0.1,4.5) node[left] {$$};
\draw[-, black!100, thin,densely dotted]  (10,0.5) -- (-0.1,0.5) node[left] {$$};
\draw[-, black!100, thin]  (10,2) -- (-0.1,2) node[left] {$20$};
\draw[-, black!100, thin]  (10,4) -- (-0.1,4) node[left] {$40$}; 
 
\draw[fill=blue, opacity=0.4] (1,0) rectangle (1.8,4.4) ;
\node[rotate=90] at (1.4,1) {2016};
\draw[fill=blue, opacity=0.6] (2,0) rectangle (2.8,4.8) ;
\node[rotate=90] at (2.4,1) {2020};
\draw[fill=red, opacity=0.4] (4,0) rectangle (4.8,3.6) ;
\node[rotate=90] at (4.4,1) {2016};
\draw[fill=red, opacity=0.6] (5,0) rectangle (5.8,2.1) ;
\node[rotate=90] at (5.4,1) {2020};
\draw[fill=orange, opacity=0.4] (7,0) rectangle (7.8,1.6) ;
\node[rotate=90] at (7.4,1) {2016};
\draw[fill=orange, opacity=0.6] (8,0) rectangle (8.8,2.4) ;
\node[rotate=90] at (8.4,1) {2020};

	\end{tikzpicture}
\end{center}



    \begin{tabular}{ |c||c|c|c|  }\hline
     Behauptung & Wahr  & Falsch & Nicht bestimmbar 		 \\ \hline\hline
    $2016$ hat die Partei $B$ mehr als doppelt so   &  \checkmark  &     &     		 \\  
     viele Stimmen wie die Partei $C$ bekommen.  &    &     &     		 \\ \hline
    Die Partei $A$ bekam $2020$ mehr Stimmen als $2016$.   &    &     &   \checkmark  		 \\ \hline
    $2020$  hat die Partei $A$ weniger Stimmen als   &    &  \checkmark   &     		 \\  
    die Parteien $B$ und $C$ zusammen bekommen.  &    &     &     		 \\ \hline
    Die Wahlen scheinen den Zahlen nach g�ltig zu sein.   &  \checkmark  &     &     		 \\ \hline 
    \end{tabular}


\begin{enumerate}
	\item Behauptung: Die Aussage ist wahr, da $B$ $36\%$ der Stimmen geholt hat, w�hrend $C$ $2016$ nur $16\%$ der Stimmen bekam. 
	
	\item Behauptung: Die Aussage kann nicht �berpr�ft werden, da es sich um relative Zahlen handelt, die sich immer auf die jeweilige Wahlbeteiligung bezieht, die $2020$ deutlich niedriger aber auch h�her gewesen sein kann.
	
	\item Behauptung: Die Aussage ist falsch, da $A$ $48\%$,  $B$ $21\%$ und $C$ $24\%$ bekam. 
	
	\item Behauptung: Die Aussage ist wahr, da die Summe der betrachteten Stimmanteile $2016$ $96\%$ und $2020$ $93\%$ betr�gt und so kleiner als $100\%$ ist. Es muss kleiner sein, da es immer ung�ltige Stimmen geben kann oder aber auch kleinere Parteien stimmen bekommen, die nicht im Diagramm dargestellt wurden.
\end{enumerate}


 

 























\end{document}